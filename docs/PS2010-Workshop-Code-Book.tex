% Options for packages loaded elsewhere
\PassOptionsToPackage{unicode}{hyperref}
\PassOptionsToPackage{hyphens}{url}
\documentclass[
]{book}
\usepackage{xcolor}
\usepackage{amsmath,amssymb}
\setcounter{secnumdepth}{5}
\usepackage{iftex}
\ifPDFTeX
  \usepackage[T1]{fontenc}
  \usepackage[utf8]{inputenc}
  \usepackage{textcomp} % provide euro and other symbols
\else % if luatex or xetex
  \usepackage{unicode-math} % this also loads fontspec
  \defaultfontfeatures{Scale=MatchLowercase}
  \defaultfontfeatures[\rmfamily]{Ligatures=TeX,Scale=1}
\fi
\usepackage{lmodern}
\ifPDFTeX\else
  % xetex/luatex font selection
\fi
% Use upquote if available, for straight quotes in verbatim environments
\IfFileExists{upquote.sty}{\usepackage{upquote}}{}
\IfFileExists{microtype.sty}{% use microtype if available
  \usepackage[]{microtype}
  \UseMicrotypeSet[protrusion]{basicmath} % disable protrusion for tt fonts
}{}
\makeatletter
\@ifundefined{KOMAClassName}{% if non-KOMA class
  \IfFileExists{parskip.sty}{%
    \usepackage{parskip}
  }{% else
    \setlength{\parindent}{0pt}
    \setlength{\parskip}{6pt plus 2pt minus 1pt}}
}{% if KOMA class
  \KOMAoptions{parskip=half}}
\makeatother
\usepackage{color}
\usepackage{fancyvrb}
\newcommand{\VerbBar}{|}
\newcommand{\VERB}{\Verb[commandchars=\\\{\}]}
\DefineVerbatimEnvironment{Highlighting}{Verbatim}{commandchars=\\\{\}}
% Add ',fontsize=\small' for more characters per line
\usepackage{framed}
\definecolor{shadecolor}{RGB}{248,248,248}
\newenvironment{Shaded}{\begin{snugshade}}{\end{snugshade}}
\newcommand{\AlertTok}[1]{\textcolor[rgb]{0.94,0.16,0.16}{#1}}
\newcommand{\AnnotationTok}[1]{\textcolor[rgb]{0.56,0.35,0.01}{\textbf{\textit{#1}}}}
\newcommand{\AttributeTok}[1]{\textcolor[rgb]{0.13,0.29,0.53}{#1}}
\newcommand{\BaseNTok}[1]{\textcolor[rgb]{0.00,0.00,0.81}{#1}}
\newcommand{\BuiltInTok}[1]{#1}
\newcommand{\CharTok}[1]{\textcolor[rgb]{0.31,0.60,0.02}{#1}}
\newcommand{\CommentTok}[1]{\textcolor[rgb]{0.56,0.35,0.01}{\textit{#1}}}
\newcommand{\CommentVarTok}[1]{\textcolor[rgb]{0.56,0.35,0.01}{\textbf{\textit{#1}}}}
\newcommand{\ConstantTok}[1]{\textcolor[rgb]{0.56,0.35,0.01}{#1}}
\newcommand{\ControlFlowTok}[1]{\textcolor[rgb]{0.13,0.29,0.53}{\textbf{#1}}}
\newcommand{\DataTypeTok}[1]{\textcolor[rgb]{0.13,0.29,0.53}{#1}}
\newcommand{\DecValTok}[1]{\textcolor[rgb]{0.00,0.00,0.81}{#1}}
\newcommand{\DocumentationTok}[1]{\textcolor[rgb]{0.56,0.35,0.01}{\textbf{\textit{#1}}}}
\newcommand{\ErrorTok}[1]{\textcolor[rgb]{0.64,0.00,0.00}{\textbf{#1}}}
\newcommand{\ExtensionTok}[1]{#1}
\newcommand{\FloatTok}[1]{\textcolor[rgb]{0.00,0.00,0.81}{#1}}
\newcommand{\FunctionTok}[1]{\textcolor[rgb]{0.13,0.29,0.53}{\textbf{#1}}}
\newcommand{\ImportTok}[1]{#1}
\newcommand{\InformationTok}[1]{\textcolor[rgb]{0.56,0.35,0.01}{\textbf{\textit{#1}}}}
\newcommand{\KeywordTok}[1]{\textcolor[rgb]{0.13,0.29,0.53}{\textbf{#1}}}
\newcommand{\NormalTok}[1]{#1}
\newcommand{\OperatorTok}[1]{\textcolor[rgb]{0.81,0.36,0.00}{\textbf{#1}}}
\newcommand{\OtherTok}[1]{\textcolor[rgb]{0.56,0.35,0.01}{#1}}
\newcommand{\PreprocessorTok}[1]{\textcolor[rgb]{0.56,0.35,0.01}{\textit{#1}}}
\newcommand{\RegionMarkerTok}[1]{#1}
\newcommand{\SpecialCharTok}[1]{\textcolor[rgb]{0.81,0.36,0.00}{\textbf{#1}}}
\newcommand{\SpecialStringTok}[1]{\textcolor[rgb]{0.31,0.60,0.02}{#1}}
\newcommand{\StringTok}[1]{\textcolor[rgb]{0.31,0.60,0.02}{#1}}
\newcommand{\VariableTok}[1]{\textcolor[rgb]{0.00,0.00,0.00}{#1}}
\newcommand{\VerbatimStringTok}[1]{\textcolor[rgb]{0.31,0.60,0.02}{#1}}
\newcommand{\WarningTok}[1]{\textcolor[rgb]{0.56,0.35,0.01}{\textbf{\textit{#1}}}}
\usepackage{longtable,booktabs,array}
\usepackage{calc} % for calculating minipage widths
% Correct order of tables after \paragraph or \subparagraph
\usepackage{etoolbox}
\makeatletter
\patchcmd\longtable{\par}{\if@noskipsec\mbox{}\fi\par}{}{}
\makeatother
% Allow footnotes in longtable head/foot
\IfFileExists{footnotehyper.sty}{\usepackage{footnotehyper}}{\usepackage{footnote}}
\makesavenoteenv{longtable}
\usepackage{graphicx}
\makeatletter
\newsavebox\pandoc@box
\newcommand*\pandocbounded[1]{% scales image to fit in text height/width
  \sbox\pandoc@box{#1}%
  \Gscale@div\@tempa{\textheight}{\dimexpr\ht\pandoc@box+\dp\pandoc@box\relax}%
  \Gscale@div\@tempb{\linewidth}{\wd\pandoc@box}%
  \ifdim\@tempb\p@<\@tempa\p@\let\@tempa\@tempb\fi% select the smaller of both
  \ifdim\@tempa\p@<\p@\scalebox{\@tempa}{\usebox\pandoc@box}%
  \else\usebox{\pandoc@box}%
  \fi%
}
% Set default figure placement to htbp
\def\fps@figure{htbp}
\makeatother
\setlength{\emergencystretch}{3em} % prevent overfull lines
\providecommand{\tightlist}{%
  \setlength{\itemsep}{0pt}\setlength{\parskip}{0pt}}
\usepackage[]{natbib}
\bibliographystyle{apalike}
\usepackage{booktabs}  
\usepackage{float}     
\let\cleardoublepage\clearpage  

% Fix table of contents blank pages
\usepackage{etoolbox}
\makeatletter
\patchcmd{\tableofcontents}{\cleardoublepage}{\clearpage}{}{}
\makeatother

% Ensure headings do not appear alone at the bottom of a page
\usepackage{titlesec}
\usepackage{needspace}
\usepackage{lmodern}  

% Keep section headings with at least 5 lines of content
\let\oldsection\section
\renewcommand{\section}{\needspace{5\baselineskip}\oldsection}

\let\oldsubsection\subsection
\renewcommand{\subsection}{\needspace{3\baselineskip}\oldsubsection}  % Move only if less than 3 lines fit

\let\oldsubsubsection\subsubsection
\renewcommand{\subsubsection}{\needspace{2\baselineskip}\oldsubsubsection}  % Smallest headings need 2 lines

% Prevents 'titlesec' errors if the package is missing
\ifdefined\titleformat
  % titlesec is loaded
\else
  \usepackage{titlesec}
\fi

\usepackage{fancyhdr}
\pagestyle{plain}  % Removes headers, keeps page numbers at the bottom
\usepackage{bookmark}
\IfFileExists{xurl.sty}{\usepackage{xurl}}{} % add URL line breaks if available
\urlstyle{same}
\hypersetup{
  pdftitle={PS2010 Workshop Code Book},
  pdfauthor={Luke Kendrick},
  hidelinks,
  pdfcreator={LaTeX via pandoc}}

\title{PS2010 Workshop Code Book}
\author{Luke Kendrick}
\date{2025-10-09}

\begin{document}
\maketitle

{
\setcounter{tocdepth}{1}
\tableofcontents
}
\chapter*{Preface}\label{preface}
\addcontentsline{toc}{chapter}{Preface}

\begin{quote}
This is the PS2010 Psychological Research Methods and Analysis Workshop Codebook.
\end{quote}

\section{About this Code Book}\label{about-this-code-book}

This code book contains information, exercises, and code for the PS2010 workshop sessions.

This is resource is a work in progress, and we're continually updating and improving it.

If you spot an error or something that doesn't look quite right, please get in touch:

\href{mailto:luke.kendrick@rhul.ac.uk}{\nolinkurl{luke.kendrick@rhul.ac.uk}}.

\section{\texorpdfstring{\textbf{License}}{License}}\label{license}

This work is licensed under Creative Commons Attribution-ShareAlike 4.0 International License \href{https://creativecommons.org/licenses/by-sa/4.0/}{(CC-BY-SA 4.0)}. You are free to share and adapt the material for non-commercial purposes, with appropriate credit and under the same license. If you adapt the material, you must distribute your contributions under the same license.

\section{\texorpdfstring{\textbf{Citation}}{Citation}}\label{citation}

Kendrick, L. T. (2025). PS2010 Workshop Code Book (Version 1.0). \url{https://luke-kendrick.github.io/r_codebook}

\chapter{Workshop 1: Data Handling Skills}\label{workshop-1-data-handling-skills}

\textbf{Aims:}

\begin{itemize}
\item
  Practice importing a .csv data file into RStudio using \texttt{read\_csv()}
\item
  Practice inspecting your data in RStudio.
\item
  Use different data wrangling functions to develop your data handling skills.
\item
  Check basic summary statistics.
\end{itemize}

\section{Exercise 1: Import the Data}\label{exercise-1-import-the-data}

\textbf{Import the Data File \texttt{guess\_who.csv}}

Before you begin, you will need the tidyverse package loaded.

\begin{Shaded}
\begin{Highlighting}[]
\FunctionTok{install.packages}\NormalTok{(}\StringTok{"tidyverse"}\NormalTok{) }\CommentTok{\#install tidyverse if you do not have it.}
\FunctionTok{library}\NormalTok{(tidyverse) }\CommentTok{\#loads tidyverse.}
\end{Highlighting}
\end{Shaded}

Next import the data file and store it as an object called \texttt{dataset}.

\begin{Shaded}
\begin{Highlighting}[]
\NormalTok{dataset }\OtherTok{\textless{}{-}} \FunctionTok{read\_csv}\NormalTok{(}\StringTok{"guess\_who.csv"}\NormalTok{)}
\end{Highlighting}
\end{Shaded}

If you see an error saying cannot find function \texttt{read\_csv()} this usually means you have not loaded (or installed) the tidyverse package.

\section{Exercise 2: Inspect and Check Your Data}\label{exercise-2-inspect-and-check-your-data}

\textbf{Take a look at your newly imported data file}

Check the top right panel (the environment) and also use the code below to inspect your data set.

\begin{Shaded}
\begin{Highlighting}[]
\FunctionTok{view}\NormalTok{(dataset) }\CommentTok{\# this will open the data in a new tab.}
\FunctionTok{names}\NormalTok{(dataset) }\CommentTok{\# this will show the variable names.}
\end{Highlighting}
\end{Shaded}

It is really important to look at the variable names as you'll be using them in code later on.

Answer Question 2.1 - 2.2 on your worksheet.

\section{Exercise 3: Change a Variable Name}\label{exercise-3-change-a-variable-name}

\textbf{One of the variable names is quite long. This can be annoying if we have to keep typing it out}

Change the variable name \texttt{do\_you\_own\_a\_pet} to \texttt{pet}. The \texttt{rename()} function will let you rename a variable.

\begin{Shaded}
\begin{Highlighting}[]
\NormalTok{dataset }\OtherTok{\textless{}{-}}\NormalTok{ dataset }\SpecialCharTok{\%\textgreater{}\%}
  \FunctionTok{rename}\NormalTok{(}\AttributeTok{pet =}\NormalTok{ do\_you\_own\_a\_pet)}
\end{Highlighting}
\end{Shaded}

Check it has worked:

\begin{Shaded}
\begin{Highlighting}[]
\FunctionTok{names}\NormalTok{(dataset) }\CommentTok{\# ask for the variable names again}
\end{Highlighting}
\end{Shaded}

\section{Exercise 4: Remove a Variable}\label{exercise-4-remove-a-variable}

\textbf{We do not really care about the \texttt{age} variable for the next few exercises.}

Let's remove it.

The code below will create a new object (once we start removing things, it is best to keep the original data file called \texttt{dataset} in the environment)

\begin{Shaded}
\begin{Highlighting}[]
\NormalTok{mydata }\OtherTok{\textless{}{-}}\NormalTok{ dataset }\SpecialCharTok{\%\textgreater{}\%}
  \FunctionTok{select}\NormalTok{(}\SpecialCharTok{{-}}\NormalTok{age)}
\end{Highlighting}
\end{Shaded}

This code will:

\begin{itemize}
\item
  Create a new object called \texttt{mydata}.
\item
  Take our original data called \texttt{dataset}
\item
  ``And then'' \texttt{\%\textgreater{}\%}
\item
  Use the select() function to remove \texttt{age} by placing a minus symbol \texttt{-} in front of it.
\end{itemize}

From now on, we will use the object called \texttt{mydata} and not the original data set.

\section{Exercise 5: Filter Cases}\label{exercise-5-filter-cases}

\textbf{We can select particular cases in our data set}

For example, I could ask: how many people were from the city of Birmingham using the code below:

\begin{Shaded}
\begin{Highlighting}[]
\NormalTok{mydata }\SpecialCharTok{\%\textgreater{}\%}
  \FunctionTok{filter}\NormalTok{(city }\SpecialCharTok{==} \StringTok{"birmingham"}\NormalTok{) }\SpecialCharTok{\%\textgreater{}\%}
  \FunctionTok{count}\NormalTok{()}
\end{Highlighting}
\end{Shaded}

Check the console (bottom left panel) for the answer.

This code will:

\begin{itemize}
\item
  Take \texttt{mydata} and then\ldots{}
\item
  Filter it by the \texttt{city} variable.
\item
  We use a double equals symbol \texttt{==} to specify an exact match.
\item
  I've added ``birmingham'' in speech marks. Note it is lowercase as to match the data set and then\ldots{}
\item
  count() the number of data points.
\end{itemize}

Adapt the code above to answer question 5.1 on the worksheet.

\section{Exercise 6: Guess Who?}\label{exercise-6-guess-who}

\textbf{We can filter based on multiple criteria}.

The code below will show us someone who is from Brighton, has a dog, and does not drink coffee.

\begin{Shaded}
\begin{Highlighting}[]
\NormalTok{mydata }\SpecialCharTok{\%\textgreater{}\%}
  \FunctionTok{filter}\NormalTok{(city }\SpecialCharTok{==} \StringTok{"brighton"}\NormalTok{, pet }\SpecialCharTok{==} \StringTok{"dog"}\NormalTok{, coffee }\SpecialCharTok{==} \DecValTok{0}\NormalTok{)}
\end{Highlighting}
\end{Shaded}

We can also use less than/more than symbols to filter data, For example, this will show all people who have a maths enjoyment score of less than 20:

\begin{Shaded}
\begin{Highlighting}[]
\NormalTok{mydata }\SpecialCharTok{\%\textgreater{}\%}
  \FunctionTok{filter}\NormalTok{(maths }\SpecialCharTok{\textless{}} \DecValTok{20}\NormalTok{)}
\end{Highlighting}
\end{Shaded}

Use what you have learned above and adapt your code to play GUESS WHO? and complete questions 6.1-6.2 on the worksheet.

\section{Exercise 7: Create a New Variable}\label{exercise-7-create-a-new-variable}

\textbf{Sometimes we might want to compute new scores or variables}

Add up the three enjoyment scores for \texttt{maths}, \texttt{science}, and \texttt{art} to create an overall score called \texttt{total\_score}.

\begin{Shaded}
\begin{Highlighting}[]
\NormalTok{mydata }\OtherTok{\textless{}{-}}\NormalTok{ mydata }\SpecialCharTok{\%\textgreater{}\%}
  \FunctionTok{mutate}\NormalTok{(}\AttributeTok{total\_score =}\NormalTok{ maths }\SpecialCharTok{+}\NormalTok{ science }\SpecialCharTok{+}\NormalTok{ art)}
\end{Highlighting}
\end{Shaded}

This code will:

\begin{itemize}
\item
  Take \texttt{mydata} to overwrite it (ready to add the new variable) and then\ldots{}
\item
  Use the \texttt{mutate()} function to create a new variable named \texttt{total\_score} which should equal \texttt{=} \texttt{maths} \texttt{+} \texttt{science} \texttt{+} \texttt{art}.
\end{itemize}

View the data set and look for the new column to see it has worked.

\begin{Shaded}
\begin{Highlighting}[]
\FunctionTok{view}\NormalTok{(mydata)}
\end{Highlighting}
\end{Shaded}

Now, we can look at who had the highest and lowest total enjoyment score.

\texttt{slice\_min} will find the row which has the lowest score:

\begin{Shaded}
\begin{Highlighting}[]
\NormalTok{mydata }\SpecialCharTok{\%\textgreater{}\%}
  \FunctionTok{slice\_min}\NormalTok{(total\_score)}
\end{Highlighting}
\end{Shaded}

\texttt{slice\_max} will find the row which has the highest score:

\begin{Shaded}
\begin{Highlighting}[]
\NormalTok{mydata }\SpecialCharTok{\%\textgreater{}\%}
  \FunctionTok{slice\_max}\NormalTok{(total\_score)}
\end{Highlighting}
\end{Shaded}

Answer questions 7.1-7.2 on the worksheet.

\section{Exercise 8: Counting and Removing Missing Data}\label{exercise-8-counting-and-removing-missing-data}

\textbf{Real data sets often are missing data points}

Different people have differing views on how to treat missing data points. For today, we will just identify and remove any. If you view the data, you might notice there are some blanks for \texttt{degree} as not everyone is studying for one.

\begin{Shaded}
\begin{Highlighting}[]
\FunctionTok{sum}\NormalTok{(}\FunctionTok{is.na}\NormalTok{(mydata}\SpecialCharTok{$}\NormalTok{degree))}
\end{Highlighting}
\end{Shaded}

This code will:

\begin{itemize}
\item
  Calculate the total number using \texttt{sum()} of\ldots{}
\item
  Any missing data points (R calls these \texttt{is.na})
\item
  We can then direct to a particular column using \texttt{mydata\$degree}. This essentially means ``look in \texttt{mydata} and then the column called \texttt{degree}. We use the dollar sign \texttt{\$} to specify the column.
\end{itemize}

If we want to remove them, we can use \texttt{filter()} again!

\begin{Shaded}
\begin{Highlighting}[]
\NormalTok{mydata }\OtherTok{\textless{}{-}}\NormalTok{ mydata }\SpecialCharTok{\%\textgreater{}\%} 
  \FunctionTok{filter}\NormalTok{(}\SpecialCharTok{!}\FunctionTok{is.na}\NormalTok{(degree))}
\end{Highlighting}
\end{Shaded}

Note: this will overwrite \texttt{mydata} and remove the cases.

Answer question 8.1 on the worksheet.

\section{Exercise 9: Summary Statistics}\label{exercise-9-summary-statistics}

\textbf{We might want to know What was the average enjoyment score?}

We can use this code to look across the data set as a whole:

\begin{Shaded}
\begin{Highlighting}[]
\FunctionTok{summary}\NormalTok{(mydata)}
\end{Highlighting}
\end{Shaded}

Look through the output in the console (bottom left panel) and answer questions 9.1-9.3 on the worksheet.

\section{Exercise 10: Fixing Luke's Broken Code}\label{exercise-10-fixing-lukes-broken-code}

Help!! My code below is not working. I need your help to fix it\ldots{}

Fix the code below to work out how many coffees were drunk by the person from canterbury and is studying medicine. Try running the code first and then work out why it doesn't work!

\begin{Shaded}
\begin{Highlighting}[]
\NormalTok{mydata }\SpecialCharTok{\%\textgreater{}\%}
  \FunctionTok{filter}\NormalTok{(}\AttributeTok{city =} \StringTok{"canterburY"}\NormalTok{, degree }\SpecialCharTok{==}\NormalTok{ medicine)}
\end{Highlighting}
\end{Shaded}

Fix the code below to work out the name of the person who is from London, has a hamster, studies psychology, and did not drink coffee. Try running the code first and then work out why it doesn't work!

\begin{Shaded}
\begin{Highlighting}[]
\NormalTok{mydata }\OtherTok{=}
  \FunctionTok{filter}\NormalTok{(city }\SpecialCharTok{==} \StringTok{"London"}\NormalTok{, pet }\SpecialCharTok{==} \StringTok{"hamsta"}\NormalTok{, degree }\SpecialCharTok{==} \StringTok{"psychology"}\NormalTok{, coffee }\SpecialCharTok{\textgreater{}}\DecValTok{1}\NormalTok{)}
\end{Highlighting}
\end{Shaded}

Answer questions 10.1-10.2 on the worksheet.

If you get stuck, use the hints below.

\begin{center}\rule{0.5\linewidth}{0.5pt}\end{center}

👀 Click for a hint

\begin{itemize}
\item
  Check for spelling errors: there are two of them.
\item
  Make sure to use double equals when specifying a label \texttt{==}.
\item
  Use quote marks when necessary. Some are missing.
\item
  Code is case sensitive. There is a capital letter where there shouldn't be one.
\item
  Use symbols correctly. We want to use the pipe \%\textgreater\% before we filter.
\item
  Use symbols correctly. More than \texttt{\textgreater{}} is not the same as \texttt{\textless{}}.
\end{itemize}

\begin{center}\rule{0.5\linewidth}{0.5pt}\end{center}

\begin{center}\rule{0.5\linewidth}{0.5pt}\end{center}

\textbf{Well Done. You have reached the end of the workshop.}

\begin{center}\rule{0.5\linewidth}{0.5pt}\end{center}

\chapter{Workshop 2: Summarising and Describing Data}\label{workshop-2-summarising-and-describing-data}

\textbf{Aims:}

\begin{itemize}
\item
  Practice importing a .csv data file into RStudio using \texttt{read\_csv()}
\item
  Practice inspecting your data in RStudio.
\item
  Calculate mean and standard deviation using the \texttt{group\_by()} and \texttt{summarise()} functions.
\item
  Visually inspect data using plots and describe data distributions.
\end{itemize}

\section{Exercise 1: Import the Data}\label{exercise-1-import-the-data-1}

\textbf{Import the Data File:} \texttt{stroop.csv}

Before you begin, you will need the tidyverse package loaded.

\begin{Shaded}
\begin{Highlighting}[]
\FunctionTok{install.packages}\NormalTok{(}\StringTok{"tidyverse"}\NormalTok{) }\CommentTok{\#install tidyverse if you do not have it.}
\FunctionTok{library}\NormalTok{(tidyverse) }\CommentTok{\#loads tidyverse.}
\end{Highlighting}
\end{Shaded}

Next import the data file and store it as an object called \texttt{dataset}.

\begin{Shaded}
\begin{Highlighting}[]
\NormalTok{dataset }\OtherTok{\textless{}{-}} \FunctionTok{read\_csv}\NormalTok{(}\StringTok{"stroop.csv"}\NormalTok{)}
\end{Highlighting}
\end{Shaded}

If you see an error saying cannot find function \texttt{read\_csv()} this usually means you have not loaded (or installed) the tidyverse package.

\section{Exercise 2: Inspect and Check Your Data}\label{exercise-2-inspect-and-check-your-data-1}

\textbf{Take a look at your newly imported data file}

Check the top right panel (the environment) and also use the code below to inspect your data set.

\begin{Shaded}
\begin{Highlighting}[]
\FunctionTok{view}\NormalTok{(dataset) }\CommentTok{\# this will open the data in a new tab.}
\FunctionTok{names}\NormalTok{(dataset) }\CommentTok{\# this will show the variable names.}
\end{Highlighting}
\end{Shaded}

It is really important to look at the variable names as you'll be using them in code later on.

\section{Exercise 3: Calculate the Stroop Inteference Score}\label{exercise-3-calculate-the-stroop-inteference-score}

\textbf{Sometimes we might want to compute new scores or variables}

Calculate the Stroop interference measure. This should be the difference between the incongruent and congruent conditions. Take the \texttt{incongruent} reaction times and then subtract the \texttt{congruent} reaction times using the code below:

\begin{Shaded}
\begin{Highlighting}[]
\NormalTok{mydata }\OtherTok{\textless{}{-}}\NormalTok{ dataset }\SpecialCharTok{\%\textgreater{}\%}
  \FunctionTok{mutate}\NormalTok{(}\AttributeTok{int =}\NormalTok{ incongruent }\SpecialCharTok{{-}}\NormalTok{ congruent)}
\end{Highlighting}
\end{Shaded}

This code will:

\begin{itemize}
\item
  Create an object called \texttt{mydata} before using the original \texttt{dataset} and then\ldots{}
\item
  Use the \texttt{mutate()} function to create a new variable named \texttt{int} which should equal \texttt{=} \texttt{incongruent} \texttt{-} (minus) \texttt{congruent}.
\end{itemize}

View the data set and look for the new column to see it has worked. Note: check the final column to see if \texttt{int} has appeared.

\begin{Shaded}
\begin{Highlighting}[]
\FunctionTok{view}\NormalTok{(mydata)}
\end{Highlighting}
\end{Shaded}

What exactly is this Stroop Interference thingy? If you want to learn more, see below.

\begin{center}\rule{0.5\linewidth}{0.5pt}\end{center}

👀 Click for more information

The interference measure in milliseconds (msecs) is the amount of extra time it took a participant to answer the incongruent (trickier trials because the colours do not match the word) compared to the congruent trials (easier trial because the colours do match the work).

In a sense, it is how many milliseconds slower you are because you need to focus your attention and engage executive functions to fight the urge to read the word rather than name the colour.

A smaller number is perhaps indicative of having better attention/executive functioning processes!

\begin{center}\rule{0.5\linewidth}{0.5pt}\end{center}

Answer questions 3.1-3.2 on the worksheet.

\section{Exercise 4: Calculate Descriptive Statistics}\label{exercise-4-calculate-descriptive-statistics}

Just as shown in the lecture, use the code below to calculate the mean and standard deviation for Stroop interference or \texttt{int}. Remember, we want to use \texttt{int} that was calculated in exercise 3.

\begin{Shaded}
\begin{Highlighting}[]
\NormalTok{desc }\OtherTok{\textless{}{-}}\NormalTok{  mydata }\SpecialCharTok{\%\textgreater{}\%}
  \FunctionTok{group\_by}\NormalTok{(NULL1) }\SpecialCharTok{\%\textgreater{}\%}
  \FunctionTok{summarise}\NormalTok{(}\AttributeTok{mean\_int =} \FunctionTok{mean}\NormalTok{(NULL2),}
            \AttributeTok{sd\_int =} \FunctionTok{sd}\NormalTok{(NULL2))}
\end{Highlighting}
\end{Shaded}

You will need to change \texttt{NULL} to match your data set. Try and give this a go on your own first, but if you aren't sure look below for help.

Think about:

\begin{itemize}
\item
  For \texttt{NULL1}: What is the name of the variable you will split the data file by (e.g., what is the grouping variable/independent variable called in \texttt{mydata})
\item
  For \texttt{NULL2}: What is the name of the score that you want to find the mean and standard deviation for (e.g., what is the dependent variable called in \texttt{mydata})
\end{itemize}

\begin{center}\rule{0.5\linewidth}{0.5pt}\end{center}

👀 Click for a hint

\begin{Shaded}
\begin{Highlighting}[]
\NormalTok{desc }\OtherTok{\textless{}{-}}\NormalTok{  mydata }\SpecialCharTok{\%\textgreater{}\%}
  \FunctionTok{group\_by}\NormalTok{(drink) }\SpecialCharTok{\%\textgreater{}\%}
  \FunctionTok{summarise}\NormalTok{(}\AttributeTok{mean\_int =} \FunctionTok{mean}\NormalTok{(int),}
            \AttributeTok{sd\_int =} \FunctionTok{sd}\NormalTok{(int))}
\end{Highlighting}
\end{Shaded}

\begin{center}\rule{0.5\linewidth}{0.5pt}\end{center}

If you look in the environment (top right panel) you will see a new object called \texttt{desc}. This is where your descriptive statistics are stored. I called it \texttt{desc} but you can call it anything you like. It is best to keep object names short and informative. We can now view that object using the \texttt{view()} function.

\begin{Shaded}
\begin{Highlighting}[]
\FunctionTok{view}\NormalTok{(desc)}
\end{Highlighting}
\end{Shaded}

Answer questions 4.1-4.2 on the worksheet.

\section{Exercise 5: Explore Data with Plots}\label{exercise-5-explore-data-with-plots}

Generate a box plot:

\begin{Shaded}
\begin{Highlighting}[]
\FunctionTok{ggplot}\NormalTok{(mydata, }\FunctionTok{aes}\NormalTok{(}\AttributeTok{x =}\NormalTok{ drink, }\AttributeTok{y =}\NormalTok{ int)) }\SpecialCharTok{+}
  \FunctionTok{geom\_boxplot}\NormalTok{(}\AttributeTok{width =}\NormalTok{ .}\DecValTok{4}\NormalTok{)}
\end{Highlighting}
\end{Shaded}

Generate histograms:

\begin{Shaded}
\begin{Highlighting}[]
\FunctionTok{ggplot}\NormalTok{(mydata, }\FunctionTok{aes}\NormalTok{(}\AttributeTok{x =}\NormalTok{ int, }\AttributeTok{fill =}\NormalTok{ drink)) }\SpecialCharTok{+}
  \FunctionTok{geom\_histogram}\NormalTok{(}\AttributeTok{colour =} \StringTok{"black"}\NormalTok{) }\SpecialCharTok{+}
  \FunctionTok{facet\_wrap}\NormalTok{(}\SpecialCharTok{\textasciitilde{}}\NormalTok{ drink)}
\end{Highlighting}
\end{Shaded}

Generate density plots:

\begin{Shaded}
\begin{Highlighting}[]
\FunctionTok{ggplot}\NormalTok{(mydata, }\FunctionTok{aes}\NormalTok{(}\AttributeTok{x =}\NormalTok{ int, }\AttributeTok{fill =}\NormalTok{ drink)) }\SpecialCharTok{+}
  \FunctionTok{geom\_density}\NormalTok{(}\AttributeTok{alpha =}\NormalTok{ .}\DecValTok{5}\NormalTok{) }\SpecialCharTok{+}
  \FunctionTok{facet\_wrap}\NormalTok{(}\SpecialCharTok{\textasciitilde{}}\NormalTok{ drink)}
\end{Highlighting}
\end{Shaded}

Answer questions 5.1-5.2 in the worksheet.

\section{\texorpdfstring{Exercise 6: What Does \texttt{facet\_wrap()} do?}{Exercise 6: What Does facet\_wrap() do?}}\label{exercise-6-what-does-facet_wrap-do}

Re-run the density plot code, except this time delete the final line. This will show what \texttt{facet\_wrap()} does. What do you notice about the plot now?

Use this code without \texttt{facet\_wrap()}.

\begin{Shaded}
\begin{Highlighting}[]
\FunctionTok{ggplot}\NormalTok{(mydata, }\FunctionTok{aes}\NormalTok{(}\AttributeTok{x =}\NormalTok{ int, }\AttributeTok{fill =}\NormalTok{ drink)) }\SpecialCharTok{+}
  \FunctionTok{geom\_density}\NormalTok{(}\AttributeTok{alpha =}\NormalTok{ .}\DecValTok{5}\NormalTok{)}
\end{Highlighting}
\end{Shaded}

Answer question 6.1 on the worksheet.

\section{Exercise 7: Save Your Amended Data File}\label{exercise-7-save-your-amended-data-file}

Your current data file has the \texttt{int} column in, calculated in exercise 3. You need to save it so you can use it for next week's workshop. You can overwrite you original .csv file using this code, which will save today's data set.

\begin{Shaded}
\begin{Highlighting}[]
\FunctionTok{write.csv}\NormalTok{(mydata, }\StringTok{"stroop.csv"}\NormalTok{)}
\end{Highlighting}
\end{Shaded}

This means the \texttt{stroop.csv} file on your computer will be updated and ready to use next week! Make sure you know where it has saved on your computer before you leave. You will need this file next week!

\begin{center}\rule{0.5\linewidth}{0.5pt}\end{center}

\textbf{Well Done. You have reached the end of the workshop.}

\begin{center}\rule{0.5\linewidth}{0.5pt}\end{center}

\chapter{\texorpdfstring{Workshop 3: \emph{t}-tests}{Workshop 3: t-tests}}\label{workshop-3-t-tests}

\textbf{Aims:}

\begin{itemize}
\item
  Practice running and interpreting a two-sample \emph{t}-test in RStudio.
\item
  Practice running and interpreting a paired \emph{t}-test in RStudio.
\end{itemize}

\textbf{Part One}

Part one of today's workshop will involved running a two-sample t-test, which is appropriate for an independent measures design with \textbf{two} groups.

You should use the same data file from last week, as you will be aiming to investigate whether Stroop interference scores (msecs) significantly differ between the Red Cow group and the control group.

You must have completed the week 2 workshop before starting this one.

\section{Exercise 1: Import the Data}\label{exercise-1-import-the-data-2}

\textbf{Import the Data File:} \texttt{stroop.csv}

Before you begin, you will need the following packages:

\begin{Shaded}
\begin{Highlighting}[]
\FunctionTok{install.packages}\NormalTok{(}\StringTok{"tidyverse"}\NormalTok{) }\CommentTok{\#install if needed.}
\FunctionTok{install.packages}\NormalTok{(}\StringTok{"rstatix"}\NormalTok{)   }\CommentTok{\#install if needed.}
\FunctionTok{install.packages}\NormalTok{(}\StringTok{"car"}\NormalTok{)       }\CommentTok{\#install if needed.}
\FunctionTok{library}\NormalTok{(tidyverse)            }\CommentTok{\#load package.}
\FunctionTok{library}\NormalTok{(rstatix)              }\CommentTok{\#load package.}
\FunctionTok{library}\NormalTok{(car)                  }\CommentTok{\#load package.}
\end{Highlighting}
\end{Shaded}

Then import the data file. Make sure it has the \texttt{int} score that you calculated last week. If not, you will need to go back and complete workshop 2.

\begin{Shaded}
\begin{Highlighting}[]
\NormalTok{mydata }\OtherTok{\textless{}{-}} \FunctionTok{read\_csv}\NormalTok{(}\StringTok{"stroop.csv"}\NormalTok{)}
\end{Highlighting}
\end{Shaded}

\section{Exercise 2: Inspect and Check Your Data}\label{exercise-2-inspect-and-check-your-data-2}

\textbf{Take a look at your newly imported data file}

Check the top right panel (the environment) and also use the code below to inspect your data set.

\begin{Shaded}
\begin{Highlighting}[]
\FunctionTok{view}\NormalTok{(mydata) }\CommentTok{\# this will open the data in a new tab.}
\FunctionTok{names}\NormalTok{(mydata) }\CommentTok{\# this will show the variable names.}
\end{Highlighting}
\end{Shaded}

It is really important to look at the variable names as you'll be using them in code later on.

\section{Exercise 3: Check Assumptions}\label{exercise-3-check-assumptions}

\subsection{Check Heteroscedasticity with Levene's Test}\label{check-heteroscedasticity-with-levenes-test}

We do not need to do this because we will just use a Welch's \emph{t}-test instead.

If you need to justify this decision (e.g., for a 3rd Year Project), the following papers might help:

\begin{itemize}
\item
  \href{https://rips-irsp.com/articles/10.5334/irsp.82}{Delacre et al.~(2017)}
\item
  \href{https://academic.oup.com/beheco/article/17/4/688/215960}{Ruxton (2006)}
\item
  For a gentler explanation, see also \href{https://daniellakens.blogspot.com/2015/01/always-use-welchs-t-test-instead-of.html}{this blog post} from Daniel Lakens.
\end{itemize}

\subsection{Check Normality with Shapiro-Wilk Test and Histograms}\label{check-normality-with-shapiro-wilk-test-and-histograms}

We need to check normality for both groups separately. We can filter the groups:

\begin{Shaded}
\begin{Highlighting}[]
\CommentTok{\#first create a data set that contains the control group only}
\NormalTok{control }\OtherTok{\textless{}{-}}\NormalTok{ mydata }\SpecialCharTok{\%\textgreater{}\%}
  \FunctionTok{filter}\NormalTok{(drink }\SpecialCharTok{==} \StringTok{"control"}\NormalTok{)}
\CommentTok{\#then run the test}
\FunctionTok{shapiro.test}\NormalTok{(control}\SpecialCharTok{$}\NormalTok{int)}

\CommentTok{\#then create a data set that contains Red Cow drinkers only}
\NormalTok{redcow }\OtherTok{\textless{}{-}}\NormalTok{ mydata }\SpecialCharTok{\%\textgreater{}\%}
  \FunctionTok{filter}\NormalTok{(drink }\SpecialCharTok{==} \StringTok{"redcow"}\NormalTok{)}
\CommentTok{\#then run the test}
\FunctionTok{shapiro.test}\NormalTok{(redcow}\SpecialCharTok{$}\NormalTok{int)}
\end{Highlighting}
\end{Shaded}

We use a dollar sign \texttt{\$} to point R to a particular column. For example, when using \texttt{shapiro.test(redcow\$int)} you are saying to run the Shapiro Test on the \texttt{redcow} only data set and the specific column called \texttt{int}

Again, we want the \emph{p}-values to be not significant. A non-significant \emph{p}-value means the data are roughly normally distributed. If the \emph{p}-value is significant, this could be an issue as it indicates the data are not normally distributed.

When reporting the Shapiro-Wilk test, you just need to report the test statistics (\emph{w} = XX) and the \emph{p}-value. Here is an example what it could look like:

\begin{quote}
\emph{W} = 0.98, \emph{p} = .875
\end{quote}

You can also visually check the data with a quick histogram:

\begin{Shaded}
\begin{Highlighting}[]
\FunctionTok{hist}\NormalTok{(control}\SpecialCharTok{$}\NormalTok{int)}
\FunctionTok{hist}\NormalTok{(redcow}\SpecialCharTok{$}\NormalTok{int)}
\end{Highlighting}
\end{Shaded}

Check the plots panel, and use the blue arrow to switch between the two plots.

\section{\texorpdfstring{Exercise 4: Run the Two-Sample \emph{t}-Test and ask for Cohen's d}{Exercise 4: Run the Two-Sample t-Test and ask for Cohen's d}}\label{exercise-4-run-the-two-sample-t-test-and-ask-for-cohens-d}

Run the \emph{t}-test.

\begin{Shaded}
\begin{Highlighting}[]
\FunctionTok{t.test}\NormalTok{(int }\SpecialCharTok{\textasciitilde{}}\NormalTok{ drink, }\AttributeTok{data =}\NormalTok{ mydata, }\AttributeTok{var.equal =} \ConstantTok{FALSE}\NormalTok{, }\AttributeTok{alternative =} \StringTok{"two.sided"}\NormalTok{)}
\end{Highlighting}
\end{Shaded}

Ask for Cohen's d:

\begin{Shaded}
\begin{Highlighting}[]
\FunctionTok{cohens\_d}\NormalTok{(}\AttributeTok{data =}\NormalTok{ mydata, total\_sleep }\SpecialCharTok{\textasciitilde{}}\NormalTok{ energy\_drink, }\AttributeTok{var.equal =} \ConstantTok{FALSE}\NormalTok{)}
\end{Highlighting}
\end{Shaded}

Interpret your \emph{t}-test.

\begin{itemize}
\item
  Is the test significant?
\item
  What is the effect size?
\item
  If significant, how do the groups differ (e.g., use descriptive statistics to interpret the difference)?
\item
  Hint: re-use the code from last week to find the mean and standard deviation for the two groups.
\end{itemize}

Answer questions 4.1-4.3 on the worksheet.

\textbf{Part Two}

Part two of today's workshop will involved running a paired t-test, which is appropriate for a repeated measures design with \textbf{two} conditions.

Let's switch it up a bit and use a different scenario and experiment with a new data set.

Here we will look at how someone's resting heart rate (BPM: beats per minute) minute might change as a result of drinking a can of Red Cow.

In this study the resting heart rate (BPM) was measured in a group of participants both \textbf{before} and \textbf{after} consuming a can of Red Cow.

The independent variable is time point: before, after. The dependent variable is BPM.

\section{Exercise 5: Import the Data}\label{exercise-5-import-the-data}

We will call the object \texttt{dat} (short name for data)

\begin{Shaded}
\begin{Highlighting}[]
\NormalTok{dat }\OtherTok{\textless{}{-}} \FunctionTok{read\_csv}\NormalTok{(}\StringTok{"bpm.csv"}\NormalTok{)}
\end{Highlighting}
\end{Shaded}

\section{Exercise 6: Inspect and Check Your Data}\label{exercise-6-inspect-and-check-your-data}

\textbf{Take a look at your newly imported data file}

Check the top right panel (the environment) and also use the code below to inspect your data set.

\begin{Shaded}
\begin{Highlighting}[]
\FunctionTok{view}\NormalTok{(dat) }\CommentTok{\# this will open the data in a new tab.}
\FunctionTok{names}\NormalTok{(dat) }\CommentTok{\# this will show the variable names.}
\end{Highlighting}
\end{Shaded}

\section{Exercise 7: Check Assumptions}\label{exercise-7-check-assumptions}

We do need to check normality, as heteroscedasiticty does not apply to repeated measures designs. However, we need to ensure the \textbf{difference} score is normally distributed.

\begin{Shaded}
\begin{Highlighting}[]
\NormalTok{diff }\OtherTok{\textless{}{-}}\NormalTok{ dat}\SpecialCharTok{$}\NormalTok{after }\SpecialCharTok{{-}}\NormalTok{ dat}\SpecialCharTok{$}\NormalTok{before  }\CommentTok{\#this will calculate the difference score.}
\FunctionTok{shapiro.test}\NormalTok{(diff)              }\CommentTok{\# run the Shapiro test}
\FunctionTok{hist}\NormalTok{(diff)                      }\CommentTok{\# also visually inspect data}
\end{Highlighting}
\end{Shaded}

You should interpret and report this in the same way as earlier (exercise 3).

Answer questions 7.1-7.2 on the worksheet.

\section{\texorpdfstring{Exercise 8: Run the Paired \emph{t}-Test and ask for Cohen's d}{Exercise 8: Run the Paired t-Test and ask for Cohen's d}}\label{exercise-8-run-the-paired-t-test-and-ask-for-cohens-d}

A paired t-test is a little different compared to the two-sample t-test.

\begin{Shaded}
\begin{Highlighting}[]
\FunctionTok{t.test}\NormalTok{(NULL1, NULL2, }\AttributeTok{paired =} \ConstantTok{TRUE}\NormalTok{)}
\end{Highlighting}
\end{Shaded}

\begin{itemize}
\item
  Change \texttt{NULL1} to the column with the first condition.
\item
  Change \texttt{NULL2} to the column with the second condition.
\end{itemize}

(Hint: you will need to use the dollar sign \texttt{\$} to specify which data set and column, e.g., \texttt{dat\$NULL1} and \texttt{dat\$NULL2}).

Try yourself first, but if you need, check the solution below.

\begin{center}\rule{0.5\linewidth}{0.5pt}\end{center}

👀 Click for a hint

\begin{Shaded}
\begin{Highlighting}[]
\FunctionTok{t.test}\NormalTok{(dat}\SpecialCharTok{$}\NormalTok{before, dat}\SpecialCharTok{$}\NormalTok{after, }\AttributeTok{paired =} \ConstantTok{TRUE}\NormalTok{)}
\end{Highlighting}
\end{Shaded}

\begin{center}\rule{0.5\linewidth}{0.5pt}\end{center}

Annoyingly, we need to use a different package for Cohen's d for a paired t-test.

\begin{Shaded}
\begin{Highlighting}[]
\FunctionTok{install.packages}\NormalTok{(}\StringTok{"effectsize"}\NormalTok{) }\CommentTok{\#install if needed.}
\FunctionTok{library}\NormalTok{(effectsize)}
\NormalTok{effectsize}\SpecialCharTok{::}\FunctionTok{cohens\_d}\NormalTok{(dat}\SpecialCharTok{$}\NormalTok{before, dat}\SpecialCharTok{$}\NormalTok{after, }\AttributeTok{paired =} \ConstantTok{TRUE}\NormalTok{)}
\end{Highlighting}
\end{Shaded}

Answer questions 8.1-8.2 on the worksheet. Hint: you will also need the descriptive statistics (see exercise 9 below).

\section{Exercise 9: Calculate Descriptive Statistics}\label{exercise-9-calculate-descriptive-statistics}

The final thing to do is to convert the data from wide format to long format. Often with repeated measures when asking for descriptive statistics or plots, we need the data in long format.

\begin{Shaded}
\begin{Highlighting}[]
\NormalTok{longd }\OtherTok{\textless{}{-}}\NormalTok{ dat }\SpecialCharTok{\%\textgreater{}\%}
  \FunctionTok{pivot\_longer}\NormalTok{(}
\NormalTok{              before}\SpecialCharTok{:}\NormalTok{after,}
              \AttributeTok{names\_to =} \StringTok{"time"}\NormalTok{,}
              \AttributeTok{values\_to =} \StringTok{"bpm"}
\NormalTok{              )}
\end{Highlighting}
\end{Shaded}

Code explanation:

\begin{itemize}
\item
  Create a new object called \texttt{longd} where we will store the long data.
\item
  Base it on the original data set called \texttt{dat} and then \texttt{\%\textgreater{}\%}
\item
  \texttt{pivot\_longer()}
\item
  The first argument should include the columns which contain your dependent variable. In this case we will use \texttt{before:after}which will then take all and any columns from \texttt{before} through to \texttt{after} (these are the only columns so it will just take the two of them).
\item
  Next we use \texttt{names\_to\ =} to tell R what we want to call our independent variable. I have used ``time'' as it was time point for this study.
\item
  Then we use \texttt{values\_to\ =} to tell R what we want to call our dependent variable. We measure beats per minute, so I have just called this ``bpm''.
\item
  Pay careful attention to the lay out of this code. For example, notice where brackets open and close, and the placement of commas to move on the the next line.
\end{itemize}

Once you have created the new long data set called \texttt{longd} we can use it to calculate descriptive statistics. But first check it to see the difference.

\begin{Shaded}
\begin{Highlighting}[]
\FunctionTok{view}\NormalTok{(longd)}
\FunctionTok{names}\NormalTok{(longd)}
\end{Highlighting}
\end{Shaded}

Now adapt the code below to ask for the mean and standard deviation.

\begin{Shaded}
\begin{Highlighting}[]
\NormalTok{desc }\OtherTok{\textless{}{-}}\NormalTok{ longd }\SpecialCharTok{\%\textgreater{}\%}
  \FunctionTok{group\_by}\NormalTok{(NULL1) }\SpecialCharTok{\%\textgreater{}\%}
  \FunctionTok{summarise}\NormalTok{(}\AttributeTok{mean\_bpm =} \FunctionTok{mean}\NormalTok{(NULL2)}
            \AttributeTok{sd\_bpm =} \FunctionTok{sd}\NormalTok{(NULL2))}
\end{Highlighting}
\end{Shaded}

Change \texttt{NULL1} to the name of the independent variable in the \texttt{long} data file, and \texttt{NULL2} to the dependent variable. If this is tricky, use \texttt{names(longd)} if you need a reminder of the variable names and revisit your notes from last week's workshop.

Make note of the mean and standard deviation for the \texttt{before} and \texttt{after} conditions.

Try yourself first, but if you need, check the solution below.

\begin{center}\rule{0.5\linewidth}{0.5pt}\end{center}

👀 Click for a hint

\begin{Shaded}
\begin{Highlighting}[]
\NormalTok{desc }\OtherTok{\textless{}{-}}\NormalTok{ longd }\SpecialCharTok{\%\textgreater{}\%}
  \FunctionTok{group\_by}\NormalTok{(time) }\SpecialCharTok{\%\textgreater{}\%}
  \FunctionTok{summarise}\NormalTok{(}\AttributeTok{mean\_bpm =} \FunctionTok{mean}\NormalTok{(bpm)}
            \AttributeTok{sd\_bpm =} \FunctionTok{sd}\NormalTok{(bpm))}
\end{Highlighting}
\end{Shaded}

\begin{center}\rule{0.5\linewidth}{0.5pt}\end{center}

\textbf{Well Done. You have reached the end of the workshop.}

\begin{center}\rule{0.5\linewidth}{0.5pt}\end{center}

\bibliography{book.bib,packages.bib}

\end{document}
